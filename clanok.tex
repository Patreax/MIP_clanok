% Metódy inžinierskej práce

\documentclass[10pt,twoside,slovak,a4paper]{article}

\usepackage[slovak]{babel}
%\usepackage[T1]{fontenc}
\usepackage[IL2]{fontenc} % lepšia sadzba písmena Ľ než v T1
\usepackage[utf8]{inputenc}
\usepackage{graphicx}
\usepackage{url} % príkaz \url na formátovanie URL
\usepackage{hyperref} % odkazy v texte budú aktívne (pri niektorých triedach dokumentov spôsobuje posun textu)

\usepackage{cite}
%\usepackage{times}

\pagestyle{headings}

\title{Porovnanie metód modelovania webových aplikácií\thanks{Semestrálny projekt v predmete Metódy inžinierskej práce, ak. rok 2021/22, vedenie: Vladimír Mlynarovič}} % meno a priezvisko vyučujúceho na cvičeniach

\author{Patrik Tomčo\\[2pt]
	{\small Slovenská technická univerzita v Bratislave}\\
	{\small Fakulta informatiky a informačných technológií}\\
	{\small \texttt{xtomco@stuba.sk}}
	}

\date{\small 16. októbra 2021} % upravte



\begin{document}

\maketitle

\begin{abstract}
\ldots
\\Prebieha práca na abstrakte.\\
Vysvetliť problematiku. Uviesť čitateľa do problematiky. Stručne popísať článok.
%\todo{Vysvetliť problematiku. Uviesť čitateľa do problematiky. Stručne popísať článok.}
\end{abstract}


\section{Úvod}

Modelmi riadený vývoj (MDD: Model-Driven Developement) sa stáva stále viac a viac dôležitou a využívanou metódou vrámci softvérového inžinierstva. 
MDD tvrdí, že softvérové systémy musia byť vyvíjané pomocou modelov. MDD proces zvyčajne začína požiadavkovou fázou, v ktorej sa definujú požiadavkové modely, z ktorých následne vzikne jeden alebo viacero koncepčných modelov~\ref{KM}. Tie majú za úlohu popísať systém bez prihliadnutia na technologické aspekty softwéru a sú neskôr využité v analytickej fáze\cite{Valderas:MDDE}. Práve týmto modelom a metódam, v ktorých su obsiahnuté,  je venovaný tento článok. Presnejšie porovnaniu jednotlivých metód a koncepčných metód z nich pozostavujúcich. Tento článok sa taktiež zaoberá popísaním rôznych MDWE (Model-Driven Web Engineering) metód. Tieto metódy a koncepčné modely článok porovnáva z pohľadu MDD, ako aj z pohľdaju funkcionality a navigácie v rámci webových aplikácií a požiadaviek používateľa na spomenutých stránkach. Metódy, ktorým sa článok primárne venuje sú OOHDM(Obejct-Oriented Hypermedia Design Model), WSDM(Web Site Design Method), OOWS(Object Oriented Web Solutions).\\

Pre porozumenie tejto problematiky je veľmi dôležité vedieť, čo presne koncepčné modely sú. Preto tento článok začne ich popisom. Ďalej bude pokračovať následovne. Sekcia 3 prezentuje prehľad popisovaných MDWE metód a ich bližší popis. Sekcia 4 sa venuje porovaniu týchto metód a koncepčným modelom z pohľadu MDD a funkcionality a navigácie v rámci  webových aplkácií. Sekcia 5 je venovaná metódam a modelom využivajúcim MDA (Model-Driven Architecture) prístup. Sekcia 6 slúži ako sumarizácia všetkého, čo sme zistili o danej problematike a sekcia 7 poskytuje prehľad bibliografie.\\
\\
Motivujte čitateľa a vysvetlite, o čom píšete. Úvod sa väčšinou nedelí na časti.

Uveďte explicitne štruktúru článku. Tu je nejaký príklad.
Základný problém, ktorý bol naznačený v úvode, je podrobnejšie vysvetlený v časti~\ref{nejaka}.
Dôležité súvislosti sú uvedené v častiach~\ref{dolezita} a~\ref{dolezitejsia}.
Záverečné poznámky prináša časť~\ref{zaver}.

\section{Koncepčné modely} \label{KM}

Koncepčné modely webových aplikácií špecifikujú jej kompozíciu a navigáciu v nej\cite{Gkatouna:Patterns}. Kompozícia web stránky definuje, ktoré stránky tvoria hypertext a ich internú kompozíciu, ako aj možnosti používateľa na zaobchádzanie so systémom. Navigácia definuje rôzne spôsoby, ako môžu byť dané stránky navzájom prepojené linkami, ale aj zobrazenie postupnosti stránok, na základe používania zo strany používateľa, a obsahom vyobrazeným na stránke. Inými slovami, koncepčné modely webových aplikácií špecifikujú jej organizáciu jej front-end rozhrania v podobe stránok, dizajnových elementov, ktoré sú prepojené linkami an uľahčenie navigácie na web stránke a manipulovania s ňou\cite{Gkatouna:Patterns}.

\section{Prehľad MDWE metód}

\section{Porovnanie koncepčných modelov} \label{ina}

\section{Porovnanie koncepčných modelov vyžívajúcich MDA prístup}

\section{Zhrnutie}

\section{Bibliografia}


\section{Úvod1}

Toto je moj uvod\\
Prvy clanok\cite{Valderas:MDDE}. A druhy clanok\cite{Gkatouna:Patterns}.


Motivujte čitateľa a vysvetlite, o čom píšete. Úvod sa väčšinou nedelí na časti.

Uveďte explicitne štruktúru článku. Tu je nejaký príklad.
Základný problém, ktorý bol naznačený v úvode, je podrobnejšie vysvetlený v časti~\ref{nejaka}.
Dôležité súvislosti sú uvedené v častiach~\ref{dolezita} a~\ref{dolezitejsia}.
Záverečné poznámky prináša časť~\ref{zaver}.



Z obr.~\ref{f:rozhod} je všetko jasné. 

\begin{figure*}[tbh]
\centering
%\includegraphics[scale=1.0]{diagram.pdf}
Aj text môže byť prezentovaný ako obrázok. Stane sa z neho označný plávajúci objekt. Po vytvorení diagramu zrušte znak \texttt{\%} pred príkazom \verb|\includegraphics| označte tento riadok ako komentár (tiež pomocou znaku \texttt{\%}).
\caption{Rozhodujúci argument.}
\label{f:rozhod}
\end{figure*}






\section{Nejaka cast}
Základným problémom je teda\ldots{} Najprv sa pozrieme na nejaké vysvetlenie (časť~\ref{ina:nejake}), a potom na ešte nejaké (časť~\ref{ina:nejake}).\footnote{Niekedy môžete potrebovať aj poznámku pod čiarou.}

Môže sa zdať, že problém vlastne nejestvuje\cite{Coplien:MPD}, ale bolo dokázané, že to tak nie je~\cite{Czarnecki:Staged, Czarnecki:Progress}. Napriek tomu, aj dnes na webe narazíme na všelijaké pochybné názory\cite{PLP-Framework}. Dôležité veci možno \emph{zdôrazniť kurzívou}.



\subsection{Nejaké vysvetlenie} \label{ina:nejake}

Niekedy treba uviesť zoznam:

\begin{itemize}
\item jedna vec
\item druhá vec
	\begin{itemize}
	\item x
	\item y
	\end{itemize}
\end{itemize}

Ten istý zoznam, len číslovaný:

\begin{enumerate}
\item jedna vec
\item druhá vec
	\begin{enumerate}
	\item x
	\item y
	\end{enumerate}
\end{enumerate}


\subsection{Ešte nejaké vysvetlenie} \label{ina:este}

\paragraph{Veľmi dôležitá poznámka.}
Niekedy je potrebné nadpisom označiť odsek. Text pokračuje hneď za nadpisom.



\section{Dôležitá časť} \label{dolezita}




\section{Ešte dôležitejšia časť} \label{dolezitejsia}




\section{Záver} \label{zaver} % prípadne iný variant názvu



%\acknowledgement{Ak niekomu chcete poďakovať\ldots}


% týmto sa generuje zoznam literatúry z obsahu súboru literatura.bib podľa toho, na čo sa v článku odkazujete
\bibliography{literatura}
\bibliographystyle{plain} % prípadne alpha, abbrv alebo hociktorý iný
\end{document}
